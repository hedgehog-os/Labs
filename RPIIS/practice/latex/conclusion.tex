\addcontentsline{toc}{section}{Заключение}

В ходе проведённого исследования был выполнен сравнительный анализ современных языков и инструментов, применяемых для проектирования и реализации многоагентных интеллектуальных систем.

В первом разделе были подробно рассмотрены агентно-ориентированные языки программирования AgentSpeak (Jason) и GOAL. Оба языка реализуют парадигму BDI (Belief–Desire–Intention), однако различаются в архитектуре управления знаниями и целями, стиле описания поведения агентов и уровне формализации. Jason обеспечивает большую гибкость и интеграцию с Java, что делает его подходящим для симуляций и прототипирования. GOAL, в свою очередь, ориентирован на строгость логической модели и формальную верификацию поведения агентов. Основные конструкции, принципы вывода и области применения каждого языка были формализованы в SCn-коде.

Во втором разделе проанализированы два современных инструмента для разработки многоагентных систем — Jason и AutoGen. Jason представляет собой зрелую платформу с поддержкой BDI-агентов и встроенными средствами визуализации, отладки и взаимодействия с артефактами и организационными структурами. AutoGen, разработанный Microsoft Research, использует мощь языковых моделей (LLM) для координации агентов, специализации ролей и автоматизации сложных задач. Он ориентирован на построение современных LLM-приложений и взаимодействие на естественном языке. Функциональные возможности и архитектурные особенности этих инструментов также были описаны и представлены в SCn-формате.

Проведённый анализ позволил выявить ключевые различия и области применимости каждого из языков и инструментов. Полученные результаты могут быть использованы при проектировании интеллектуальных систем, требующих гибкой координации агентов, адаптивного поведения и поддержки семантической обработки знаний. Также сформулированы рекомендации по выбору технологий в зависимости от задач моделирования, уровня формализации и требований к масштабируемости и интеграции.