\begin{SCn}
\begin{small}

\scnheader{Jason}
\scnidtf{Инструмент и платформа для программирования многоагентных систем на базе языка AgentSpeak}
\scnidtf{Платформа для разработки BDI-агентов (belief-desire-intention)}
\scniselement{агентно-ориентированное программное обеспечение}
\begin{scnrelfromlist}{разработчик}
    \scnitem{Jomi F. Hübner}
    \scnitem{Rafael H. Bordini}
\end{scnrelfromlist}
\scnrelfrom{год создания}{2005}
\scntext{текущая стабильная версия}{2.5.1}
\scntext{дата выпуска текущей версии}{30 января 2024}
\scnrelfrom{лицензия}{GNU GPL v3}
\scnrelfrom{язык разработки}{Java}
\scntext{назначение}{Разработка и моделирование BDI-многоагентных систем с автономным поведением на основе логических правил и целей}
\begin{scnrelfromset}{варианты исполнения}
    \scnitem{Jason IDE}
    \begin{scnindent}
        \scnidtf{Встроенная среда разработки на базе Eclipse}
        \scnnote{Поддержка редактирования кода на AgentSpeak, управление проектами, отладка и визуализация среды}
    \end{scnindent}
    \scnitem{Jason CLI}
    \begin{scnindent}
        \scnidtf{Командная оболочка для запуска MAS-программ Jason}
        \scnnote{Позволяет исполнять агенты без GUI, подходит для серверного или тестового запуска}
    \end{scnindent}
\end{scnrelfromset}
\begin{scnrelfromset}{функциональные возможности}
    \scnfileitem{интерпретация планов BDI}
    \scnfileitem{обработка отказов и метаданные планов}
    \scnfileitem{интеграция с Java-библиотеками и пользовательскими внутренними действиями}
    \scnfileitem{визуальные инструменты отладки («mind inspector»)}
\end{scnrelfromset}

\begin{scnrelfromset}{поддерживаемые языки}
    \scnitem{AgentSpeak(L)}
    \scnitem{FIPA-ACL (через JADE/SACI)}
\end{scnrelfromset}
\begin{scnrelfromset}{экосистема расширений}
    \scnitem{Moise+}
    \scnitem{Cartago}
    \scnitem{Jason IDE}
    \scnitem{FIPA Messaging}
\end{scnrelfromset}
\begin{scnrelfromset}{преимущества}
    \scnfileitem{простота логического описания поведения агентов}
    \scnfileitem{гибкость и расширяемость за счёт Java-интеграции}
    \scnfileitem{хорошо подходит для академических и исследовательских проектов}
    \scnfileitem{открытый исходный код и активное сообщество}
    \scnfileitem{интеграция с JADE, OpenMAS и другими средами}
\end{scnrelfromset}
\begin{scnrelfromset}{недостатки}
    \scnfileitem{относительно низкий уровень абстракции среды исполнения}
    \scnfileitem{ограниченные графические средства моделирования и визуализации}
    \scnfileitem{отсутствие встроенных средств машинного обучения или генерации поведения}
\end{scnrelfromset}
\begin{scnrelfromset}{библиографические источники}
    \scnfileitem{Bordini R.H., Hübner J.F., Wooldridge M. Programming Multi-Agent Systems in AgentSpeak using Jason. Wiley, 2007.}
    \scnfileitem{Jason [Электронный ресурс] // Официальный сайт. URL: https://jason.sourceforge.net (дата обращения: 12.05.2025).}
    \scnfileitem{Bordini R.H., et al. Multi-Agent Programming: Languages, Tools and Applications. Springer, 2009.}
\end{scnrelfromset}

\scnheader{AutoGen}
\scnidtf{Платформа для создания многоагентных систем, основанных на LLM и управляемых автогенерацией действий и ролей}
\scniselement{средство для создания LLM-агентов}
\scniselement{агентно-ориентированное программное обеспечение}
\scnrelfrom{разработчик}{Microsoft Research}
\scnrelfrom{год создания}{2023}
\scnrelfrom{лицензия}{MIT}
\scnrelfrom{текущая стабильная версия}{0.2.5}
\scnrelfrom{дата выпуска текущей версии}{12 марта 2025}
\scnrelfrom{язык разработки}{Python}
\scntext{назначение}{Оркестровка агентов с LLM-ядером для автоматизации задач, требующих координации, общения и ролевой специализации}
\scntext{архитектурные особенности}{AutoGen построен на взаимодействии между агентами, каждый из которых представляет собой LLM-инстанс с определённой ролевой функцией. Поддерживаются циклы общения, делегирование, принятие решений и выполнение задач через инструменты.}
\begin{scnrelfromset}{варианты исполнения}
    \scnitem{AutoGen Core}
    \begin{scnindent}
        \scntext{примечание}{Позволяет моделировать как одиночные агенты, так и команды с контролем ролевого поведения}
    \end{scnindent}
    \scnitem{AutoGen Studio (в разработке)}
    \begin{scnindent}
        \scnidtf{Визуальная среда проектирования многоагентных сценариев}
    \end{scnindent}
\end{scnrelfromset}
\begin{scnrelfromset}{функциональные возможности}
    \scnfileitem{ролевое программирование LLM-агентов}
    \scnfileitem{создание сценариев кооперации, конкуренции, делегирования}
    \scnfileitem{поддержка инструментов (функций, API), вызываемых агентами}
    \scnfileitem{использование встроенной памяти и истории взаимодействия}
    \scnfileitem{возможность асинхронного исполнения агентов}
    \scnfileitem{интеграция с внешними API и локальными функциями через Python-интерфейсы}
    \scnfileitem{обработка запросов, генерация текстов, планирование действий}
\end{scnrelfromset}
\begin{scnrelfromset}{поддерживаемые модели}
    \scnitem{OpenAI GPT (через API)}
    \scnitem{Azure OpenAI}
    \scnitem{локальные модели через OpenAI-клонирующие интерфейсы (включая LM Studio)}
\end{scnrelfromset}
\begin{scnrelfromset}{преимущества}
    \scnfileitem{гибкость построения диалоговых и функциональных цепочек агентов}
    \scnfileitem{высокий уровень абстракции и простота настройки логики через Python}
    \scnfileitem{интеграция с современными LLM для широкого круга задач}
    \scnfileitem{поддержка автономных агентов с планированием и памятью}
    \scnfileitem{активная разработка и поддержка со стороны Microsoft Research}
\end{scnrelfromset}
\begin{scnrelfromset}{недостатки}
    \scnfileitem{зависимость от внешних LLM-платформ или ресурсов GPU}
    \scnfileitem{относительная новизна и нестабильность некоторых функций}
    \scnfileitem{отсутствие полноценной визуальной среды на момент мая 2025 года}
    \scnfileitem{отсутствие встроенного редактора ролей и сценариев}
\end{scnrelfromset}
\begin{scnrelfromset}{библиографические источники}
    \scnfileitem{AutoGen [Электронный ресурс] // GitHub repository. URL: https://github.com/microsoft/autogen (дата обращения: 12.05.2025).}
    \scnfileitem{Liu B., Wang D., Wang Y. et al. AutoGen: Enabling Next-Gen LLM Applications via Multi-Agent Conversation. arXiv preprint arXiv:2309.14610, 2023.}
    \scnfileitem{Microsoft Research Blog on AutoGen. URL: https://www.microsoft.com/en-us/research/blog (дата обращения: 12.05.2025).}
\end{scnrelfromset}

\bigskip

\scnheader{Сравнение систем Jason и AutoGen}
\begin{scneqtovector}
    \scnitem{Jason}
    \scnitem{AutoGen}
\end{scneqtovector}
\begin{scnrelfromset}{сходства}
    \scnfileitem{Поддерживают создание многоагентных систем}
    \scnfileitem{Обеспечивают автономное поведение агентов}
    \scnfileitem{Поддерживают межагентное взаимодействие и передачу сообщений}
    \scnfileitem{Реализуют программную модель агентов с внутренним состоянием и логикой поведения}
    \scnfileitem{Позволяют строить сложные сценарии взаимодействия}
    \scnfileitem{Расширяются за счёт пользовательских функций и внешних библиотек}
\end{scnrelfromset}
\begin{scnrelfromset}{различия}
    \scnfileitem{Jason ориентирован на BDI-модель и логическое программирование, AutoGen — на диалоговые LLM-модели и ролевую координацию}
    \scnfileitem{Jason использует AgentSpeak и реализован на Java, AutoGen написан на Python и использует современные LLM (например, GPT-4)}
    \scnfileitem{AutoGen предполагает наличие доступа к языковым моделям через API или локальные модели, Jason работает автономно}
    \scnfileitem{Jason фокусируется на логике действий, перцепциях и намерениях, AutoGen — на генерации ответов, стратегий и коллаборации между LLM-агентами}
    \scnfileitem{AutoGen ориентирован на автоматизацию LLM-задач (чтение, планирование, вызовы функций), Jason — на автономных программных агентов в искусственных средах}
    \scnfileitem{Jason имеет зрелую экосистему и документацию, AutoGen — активно развивающийся, но ещё формирующийся проект}
\end{scnrelfromset}
\begin{scnrelfromset}{рекомендации по выбору}
    \scnfileitem{Jason рекомендуется для учебных, исследовательских и симуляционных задач, связанных с классическими агентными моделями BDI}
    \scnfileitem{Jason рекомендуется для учебных, исследовательских и симуляционных задач, связанных с классическими агентными моделями BDI}
    \scnfileitem{AutoGen целесообразен для построения современных LLM-приложений с агентной координацией, включая ассистентов, планировщиков, агентов-аналитиков}
    \scnfileitem{AutoGen предпочтителен при необходимости в высокоуровневом управлении агентами и взаимодействии с внешними системами через языковые модели}
    \scnfileitem{Jason лучше подойдёт для реализации агентов в ограниченных средах, AutoGen — для задач, требующих гибкости, творчества и адаптивности через LLM}
\end{scnrelfromset}

\end{small}
\end{SCn}
